% LaTeX resume using res.sty
\documentstyle[fancyhead]{res}
% Use \documentstyle[fancyhead,newcent]{res} to get New Century Schoolbook
% Postscript font; the fancyhead option is used to get 2 line header
% Use \documentstyle[fancyhead]{res} to get default (Computer Modern) font
\setlength{\topmargin}{-0.6in}  % Start text higher on the page 
\setlength{\textheight}{9.8in}  % increase textheight to fit more on a page
\setlength{\headrulewidth}{0pt} % suppress line drawn by default with fancyhead
\setlength{\headsep}{0.2in}     % space between header and text
\setlength{\headheight}{12pt}   % make room for header

%\lhead{\hspace*{-\sectionwidth}Brenda Baud} % force lhead all the way left
%\rhead{Page \thepage}  % put page number at right
\cfoot{}  % the footer is empty
\pagestyle{fancy} % set pagestyle for the document
\begin{document} 

% Center the name over the entire width of resume:
\moveleft.5\hoffset\centerline{\large\bf Intai Huang}
% Draw a horizontal line the whole width of resume:
\moveleft\hoffset\vbox{\hrule width\resumewidth height 1pt}\smallskip
% address begins here
% Again, the address lines must be centered over entire width of resume:
\moveleft.5\hoffset\centerline{100A Owairaka Ave.}
\moveleft.5\hoffset\centerline{Mt. Albert, Auckland}
\vspace{0.1in}
\moveleft.5\hoffset\centerline{Mobile: (021) 107 5713}
\moveleft.5\hoffset\centerline{Email: intai.hg@gmail.com}

\begin{resume}
\vspace{0.1in}

\section{EDUCATION}
\vspace{0.1in} 
Software Education Associates Ltd,\\
Advanced C++ Programming, 8-10 June 2009

The University of Auckland,\\
Master of Science with First Class Honours,\\
in Computer Science, 2004 
 
The University of Auckland,\\
Bachelor of Technology with First Class Honours,\\
in Information Technology, 2000-2003 
 
Lynfield College, 1998-1999 
 
\section{SKILLS PROFILE} 
\vspace{0.1in}

I have 13+ years commercial experience in software development with C/C++, Objective C, PHP and JavaScript on handheld devices and web environment. I also have experience on computer graphics with OpenGL and DirectX from university and making the Cube Game on Sony PlayStation Portable platform for Metia Interactive. Programming within the hardware restrictions that applies to mobile devices was challenging and interesting.
\\\\
At Navman Technology, I experienced the importance of discipline in a large-scale software development team through coding standard, peer review and task tracking system. Agile project management also helps in enforcing discipline by involving quality assurance throughout the entire development and by reflecting and adapting to changes in shorter iterations.
\\\\
The team leader experience at Fishpond has utilised my organising and coordinating skills within the team and between departments. As a team leader: \\
\begin{itemize}
\item[$\circ$] I'm a good listener.
\item[$\circ$] Create a team environment where team members can concentrate without being distracted by rapidly changing features and ideas.
\item[$\circ$] Create an atmosphere to have the freedom to be creative. At the same time keeping the creativity coordinated as a team.
\item[$\circ$] Assign tasks out of individual team members' comfort zones gradually to build or extend skills, and make the jobs more exciting and challenging.\\
\end{itemize}

Developing the mobile consumer app and sites at GrabOne has expanded my knowledge of JavaScript, HTML5 and limitations of different mobile browsers. It used Symfony framework with Doctrine and JQuery Mobile. Also gained more experience on iOS and Android platform from making and maintaining the GrabOne merchant app.
\\\\
Developed and hosting PixPrism (http://pixprism.com) with Symfony2, MySql, Sass, RequireJs and Puppet on EC2 and S3 to create mobile friendly photo galleries of any website. Leverage the same codebase and servers for XianBridal (http://xianbridal.co.nz).
\newpage
At Findly, I have gained experience on building single page web apps with AngularJS, Bacon.js, Facebook React and Flux. Node.js is used in one of the products to proxy and transform third party websites to be mobile friendly. With all the frameworks and libraries, I have seen how difficult for company to manage products which are built on top of a young and energetic coding environment like JavaScript.
\\\\
Developed MyRadio (http://myradio.link) to manage podcasts with AngularJS, Bacon.js, Underscore.js, Traceur, ES6, Sass, Browserify, Gulp and Firebase. In the project, I experimented on using Facebook Flux dispatcher concept in AngularJS. Also focus on directive to minimise view controller to align with community's trend.
\\\\
For MyVodafone web app, I setup and started writing unit tests to improve the maintainability of the codebase. Introduced Flux partially to keep single source of truth in store with the help of functional reactive programming. As a result, directives can easily be more modular and smaller.
\\\\
Created Bdux a Flux architecture implementation out of enjoyment of Bacon.js, Redux and React. It is open sourced on GitHub and released as an npm package (https://www.npmjs.com/\%7Eintai). Bdux is reactive all the way from action to React component. It has Redux style time travel through middleware and reducer (https://github.com/Intai/bdux-timetravel). The time travel middleware includes React-Native components as well. Universal JavaScript can also be implemented through middleware (https://github.com/Intai/bdux-universal).
\\\\
The ProjectManager.com web app has fairly complex UI features for example custom spreadsheet and drag-n-drop. It's quite challenging to implement the features with a balance between maintainability and performance using React, Bacon.js and Ramda. A collection of base components was created to be the foundation to modulate the complex features. The reality of coexisting old and new codebases also adds to the complexity where iframes are used to encapsulate server side rendered pages.

\section{WORK HISTORY and EXPERIENCE} 
\vspace{0.15in} 

\begin{tabular}{ll}
\parbox[t]{30mm}{Present - \\ 2016, Sep,\\Developer\\at ProjectManager} & \parbox[t]{116mm}{

Started working at ProjectManager.com to create a new web app to blend agile project management to the currently more waterfall focused management system. Technically switching from the server side rendered JQuery codebase to a React single page app. Because a total rewrite would take too long to deliver, the process has to allow both codebases to coexist while delivering the MVP.

}\\\\
\parbox[t]{30mm}{2016, Aug - \\ 2015, May,\\Developer\\at Vodafone} & \parbox[t]{116mm}{

Worked on MyVodafone self-service web app. The JavaScript codebase is based on AngularJS 1.2 and Bootstrap 3 with a lot of JQuery plugins. The bridgings between them are usually fragile. And the wide use of global scopes and huge controllers increase the difficulty of maintenance. I removed global scopes and broke down directives to be more modular to improve readability and maintainability. 

}\\\\
\parbox[t]{30mm}{2015, Apri - \\ 2014, Jul,\\Developer\\at Findly} & \parbox[t]{116mm}{

Worked at Findly on Pollinator project to capture job applicants' details directly and CXApply to acquire applicants through other ATS (Applicant Tracking System) by providing mobile friendly version of their websites. Experienced the downside of framework like AngularJS to have complex relationships between view and models, which Facebook is trying to tackle with React and Flux.

}\\\\
\parbox[t]{30mm}{2014, Jul - \\ 2011, Jun,\\Developer\\at GrabOne} & \parbox[t]{116mm}{

Started working at GrabOne which sells online daily deals and coupons. Responsible for the iOS native wrapper and mobile sites. The native side involved Key Chain, Passbook, Reminders integration and cookies management, location and push notification. The web side used Symfony1.4 and a forked JQuery Mobile with optimisations for GrabOne. Offline capability is achieved with HTML5 application cache. Scrolling and key frame animation implemented with CSS transition and Javascript. 

}\end{tabular}\\
\begin{tabular}{ll}
\parbox[t]{30mm}{} & \parbox[t]{116mm}{

Developed GrabOne merchant app on iOS using Auto Layout, Core Data and ZBar library. The app consumes JSON response from API and handles offline capability for unstable internet connection.

}\\\\
\parbox[t]{30mm}{2011, Jun - \\ 2010, Jul,\\Team Lead\\at Fishpond} & \parbox[t]{116mm}{

Took the responsibility of leading the release team to improve stability of customer sites and internal tools. The team had three QAs and a developer who carried out release process and maintained Nagios alerts. The QAs did manual testing and automated regression tests using Selenium across different browsers. Performance test through XHProf, Pingdom and Circonus. Load test through JMeter. As the team leader, I also coordinated end user testing for internal tools with other departments. Wrote outage reports describing what went wrong, how we can prevent them from happening again and how to catch them sooner.
\\\\
The team later expanded outside release management to include three more overseas developers. It was quite a challenge to have smooth and efficient communication digitally across different time zones using email and task/bug tracking system.

}\\\\
\parbox[t]{30mm}{2010, Apr - \\ 2010, Jan,\\Developer\\at Fishpond} & \parbox[t]{116mm}{

Started working at Fishpond which is an Australasian online store selling books, music, movies, games, toys, electronics and stationery. (http://fishpond.co.nz) The sites had about 20 million products listed and 100 thousand page views per day. Developed in PHP under both Zend Framework and legacy style scripts. The customer team was responsible for backend logics for customer facing sites and Solr search indexing. Temporarily took the place of customer team lead for about two to three months due to the original team lead being promoted. The position was responsible for clarifying requirements, break down into sub tasks, provide a direction of implementation and code reviews.

}\\\\
\parbox[t]{30mm}{2010, Jan - \\ 2007, Jul,\\Developer\\at Navman} & \parbox[t]{116mm}{

Started working at Navman Technology which designs Personal Navigation Devices. Exposed to Component Object Model (COM) structure. COM structure is suitable for Navman's business model as there are teams across different sites internationally. The structure can provide robust and extensible interfaces between teams with details of implementation hidden. Exposed to Model-View-Controller (MVC) design pattern which is utilised to present and process user interactions from map display. Designed and programmed a system to animate positions or opacities of UI components which increased smoothness and playfulness which are important in consumer products.
\\\\
Worked on a newly designed UI inspired by IDEO. (http://ideo.com) Principal ideas in the design are the iPhone-like scrolling and the ability to access the map anywhere and anytime. Smoothness was also an important requirement of the design. Therefore, lots of effort was made to ensure the best performance by simplifying class inheritances and optimising the low level rendering mechanism. Other than the speed of rendering, in order to ensure the best user experience, interactions to touch screens had to be processed and tweaked differently for resistance and capacitive screen.
\\\\
Worked on Middleware project. Split application layer into Middleware and UI to be managed by different teams. On the Middleware side, the aim was to provide an object oriented library with limited functionalities to make it simple and reduce the learning curve on the UI layer. Technically that means switching from internal classes to a public library by refining the interfaces to be less powerful but more intuitive, more robust and well documented to be used by the UI layer.

}\end{tabular}\\
\begin{tabular}{ll}
\parbox[t]{30mm}{2007, June - \\ 2006, May,\\Developer\\at Metia Interactive} & \parbox[t]{116mm}{

The Cube project officially started with four programmers and four artists. The game has single player mode, two player mode (AdHoc) and a level editor. The Cube Game had been published in USA, Europe and Japan. (http://thecubegame.com)
\\\\
Programmed the Cube Game onto Windows platform using DirectX with programmable vertex and pixel shader. The Windows version includes new features like lighting and shadow to make it ready for Xbox 360.

}\\\\
\parbox[t]{30mm}{2006, Jan - Sep,\\Developer\\at Metia Interactive} & \parbox[t]{116mm}{

Programmed the demo onto PSP platform for Metia Interactive to show at the Australian Game Developers Conference (AGDC). Scheduled the Cube project on milestone basis. Wrote technical documents and provided any technical information publishers requested.

}\\\\
\parbox[t]{30mm}{2005, Feb - \\ 2004,} & \parbox[t]{116mm}{

Completed master thesis about image-based model simplification with a working application implemented using DirectX to analyse the hypothesis. Presented the paper at IVCNZ '04.

}\\\\
\parbox[t]{30mm}{2003 - 2004,} & \parbox[t]{116mm}{

Designed and programmed the SOIL software for the Structural Engineering Society of New Zealand (SESOC).\\(http://sesoc.org.nz/downloads/soils.html)

}\end{tabular} 

%\newpage
\vspace{0.15in} 

\section{PUBLICATIONS} 
\vspace{0.1in}

"Improved Billboard Clouds for Extreme Model Simplification". In-Tai Huang, Kevin Novins and Burkhard Wuensche, \textit{in Proceedings of IVCNZ '04}, Akaroa, New Zealand, 21-23 November 2004, pp. 255-260. 

"Improved Billboard Clouds for Extreme Model Simplification". In-Tai Huang, MSc thesis, Department of Computer Science, University of Auckland, 2004.

\section{REFEREES} 
\vspace{0.1in}

%Jimmy Lin\\
%Programmer\\
%Right Hemisphere\\
%Ph: 021-2607786
Available upon request.
 
\end{resume}
\end{document}






























