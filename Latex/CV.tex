% LaTeX resume using res.sty
\documentstyle[fancyhead]{res}
% Use \documentstyle[fancyhead,newcent]{res} to get New Century Schoolbook
% Postscript font; the fancyhead option is used to get 2 line header
% Use \documentstyle[fancyhead]{res} to get default (Computer Modern) font
\setlength{\topmargin}{-0.6in}  % Start text higher on the page 
\setlength{\textheight}{9.8in}  % increase textheight to fit more on a page
\setlength{\headrulewidth}{0pt} % suppress line drawn by default with fancyhead
\setlength{\headsep}{0.2in}     % space between header and text
\setlength{\headheight}{12pt}   % make room for header

%\lhead{\hspace*{-\sectionwidth}Brenda Baud} % force lhead all the way left
%\rhead{Page \thepage}  % put page number at right
\cfoot{}  % the footer is empty
\pagestyle{fancy} % set pagestyle for the document
\begin{document} 

% Center the name over the entire width of resume:
\moveleft.5\hoffset\centerline{\large\bf Intai Huang}
% Draw a horizontal line the whole width of resume:
\moveleft\hoffset\vbox{\hrule width\resumewidth height 1pt}\smallskip
% address begins here
% Again, the address lines must be centered over entire width of resume:
\moveleft.5\hoffset\centerline{100A Owairaka Ave.}
\moveleft.5\hoffset\centerline{Mt. Albert, Auckland}
\vspace{0.1in}
\moveleft.5\hoffset\centerline{Mobile: (021) 107 5713}
\moveleft.5\hoffset\centerline{Email: intai.hg@gmail.com}

\begin{resume}
\vspace{0.1in}
 
\section{SKILLS PROFILE} 
\vspace{0.1in}

I have 15+ years of commercial experience in software development with JavaScript, C/C++, Objective C and PHP on handheld devices and web environment. I also have experience on computer graphics with OpenGL and DirectX from university and making the Cube Game on Sony PlayStation Portable platform for Metia Interactive. In recent years, I'm mostly around JavaScript development.

\section{PERSONAL PROJECTS} 
\vspace{0.1in}

Developed and hosted PixPrism with Symfony2, MySql, Sass, RequireJs and Puppet on EC2 and S3 to create mobile friendly photo galleries of any website.

Developed MyRadio to manage podcasts with AngularJS, Bacon.js, Underscore.js, Traceur, ES6, Sass, Browserify, Gulp and Firebase. In the project, I experimented on using Facebook Flux dispatcher concept in AngularJS. Also focus on directive to minimise view controller.

Created Bdux a Flux architecture implementation out of enjoyment of Bacon.js, Redux and React. It is open sourced on GitHub and released as an npm package (https://www.npmjs.com/\%7Eintai). Bdux is reactive all the way from action to React component. It has Redux style time travel through middleware and reducer (https://github.com/Intai/bdux-timetravel). Universal rendering can also be implemented through middleware (https://github.com/Intai/bdux-universal).

\section{WORK HISTORY and EXPERIENCE} 
\vspace{0.15in} 

\hspace*{-0.25in} 2018 Nov - Present,\\
\hspace*{-0.25in} Senior Software Architect at ProjectManager.com

The ProjectManager.com web app has fairly complex UI features. For example custom spreadsheet, drag-n-drop and SocketIO concurrency. The new frontend features are built using React, Bacon.js and Ramda. A collection of base components are created to be the foundation of complex features. The reality of coexisting old and new codebases adds to the complexity where iframes are used to encapsulate server side rendered pages.

There are two squads with about 6 frontend focused developers and 2 backend focused. Responsible for elaborating epics and stories with product owners and design team. Slicing stories to smaller testable units and planning technically to empower developers to work in parallel without blocking each other. Providing feedbacks in pull requests as code review, and day-to-day technical discussions.

The company just released the MVP of a big re-design recently with a dark mode. Thanks to everyone's hard work and discipline, the release was smooth. Another important feature got released recently was task workflow automation, which requires risky fundamental changes in the frontend state management and SocketIO concurrency events.

I also made initiative to setup CI/CD for third party integrations using Docker, Amazon CodeBuild, ECR, ECS, CodeDeploy and CodePipeline. Wrote CloudFormation as documentation and be able to repeat the exact same setup in the future.

\newpage
\hspace*{-0.25in} 2018 May - 2018 Oct,\\
\hspace*{-0.25in} Senior Developer at Vodafone

Took a contract to work on MyVodafone self-service app in React Native. The reducers in Redux are maintainable, readable and easy to unit test. We could achieve at least 100\% test coverage in the space. Redux-Saga was used before and after reducers, which in my opinion was abused and got a bit out of control.

\hspace*{-0.25in} 2016 Sep - 2018 Apr,\\
\hspace*{-0.25in} Senior Developer at ProjectManager.com

Started working at ProjectManager.com to create a new web app for hybrid project management between agile and waterfall. Technically switching from server side rendered JQuery codebase to a React single page app. Because a complete rewrite would take too long to deliver, the process has to allow both codebases to coexist while delivering the MVP.

\hspace*{-0.25in} 2015 May - 2016 Aug,\\
\hspace*{-0.25in} Senior Developer at Vodafone

Worked on MyVodafone web app. The JavaScript codebase uses AngularJS 1.2 and Bootstrap 3 with a lot of JQuery plugins. The bridgings between them are fragile. The wide use of global scopes and huge controllers increase the difficulty of maintenance. I removed global scopes and broke down directives to be more modular to improve maintainability and readability. 

\hspace*{-0.25in} 2014 Jul - 2015 Apri,\\
\hspace*{-0.25in} Developer at Findly

Worked at Findly on Pollinator project to capture job applicants' details directly and CXApply to acquire applicants through other ATS (Applicant Tracking System) by providing mobile friendly version of their websites. The codebase suffered from the downside of framework like AngularJS to have complex relationships between view and models, which Facebook is trying to tackle with React and Flux.

\hspace*{-0.25in} 2011 Jun - 2014 Jul,\\
\hspace*{-0.25in} Developer at GrabOne

Started working at GrabOne. Responsible for the iOS native wrapper and mobile sites. The native side involved Key Chain, Passbook, Reminders integration and cookies management, location and push notification. The web side used Symfony1.4 and a forked JQuery Mobile with optimisations for GrabOne. Offline capability is achieved with HTML5 application cache. Scrolling and key frame animation implemented with CSS transition and Javascript. 

Developed GrabOne merchant app on iOS using Auto Layout, Core Data and ZBar library. The app consumes JSON response from API and handles offline capability for unstable internet connection.

\hspace*{-0.25in} 2010 Jul - 2011 Jun,\\
\hspace*{-0.25in} Team Lead at Fishpond

Took the responsibility of leading the release team to improve stability of customer sites and internal tools. The team had 3 QAs and a developer who carried out release process and maintained Nagios alerts. The QAs did manual testing and automated regression tests using Selenium across different browsers. Performance test through XHProf, Pingdom and Circonus. Load test through JMeter. As the team leader, I also coordinated end user testing for internal tools with other departments. Wrote outage reports describing what went wrong, how we can prevent them from happening again and how to catch them sooner.

The team later expanded outside release management to include 3 more overseas developers. It was quite challenging to have efficient communication digitally across different time zones using email and task/bug tracking system.

\hspace*{-0.25in} 2010 Jan - 2010 Apr,\\
\hspace*{-0.25in} Developer at Fishpond

Started working at Fishpond. The sites had about 20 million products listed and 100 thousand page views per day. Developed in PHP under both Zend Framework and legacy style scripts. The customer team was responsible for backend logics for customer facing sites and Solr search indexing. Temporarily took the place of customer team lead for about two to three months due to the original team lead being promoted. The position was responsible for clarifying requirements, breaking down into sub tasks, providing technical directions and code reviews.

\hspace*{-0.25in} 2007 Jul - 2010 Jan,\\
\hspace*{-0.25in} Developer at Navman

Worked in Navman Technology which designs Personal Navigation Devices. Exposed to Component Object Model (COM) structure which is suitable for Navman's business model as there are teams across different sites internationally. The structure can provide robust and extensible interfaces between teams with details of implementation hidden. Model-View-Controller (MVC) design pattern is utilised to present and process user interactions from map display.

Worked on a newly designed UI inspired by IDEO http://ideo.com. It has iPhone-like scrolling and the ability to access the map anywhere and anytime. Smoothness was an important requirement of the design. Therefore, lots of effort was made to ensure the best performance by simplifying class inheritances, optimising the low level rendering mechanism and tweaking separately for resistance and capacitive screen

Worked on Middleware project to provide an object oriented library with limited functionalities to reduce the learning curve on the UI layer. Technically that means refining the interfaces to be less powerful but more intuitive and well documented.

\hspace*{-0.25in} 2006 Jan- 2007 June,\\
\hspace*{-0.25in} Developer at Metia Interactive

The Cube project officially started with four programmers and four artists. The game has a single player mode, two player mode (AdHoc) and a level editor. The Cube Game had been published in USA, Europe and Japan. (http://thecubegame.com)
\\\\
Programmed the Cube Game onto Windows platform using DirectX with programmable vertex and pixel shader. The Windows version includes new features like lighting and shadow to make it ready for Xbox 360.

\hspace*{-0.25in} 2004 - 2005 Feb,

Completed master thesis about image-based model simplification with a working application implemented using DirectX to analyse the hypothesis. Presented the paper at IVCNZ '04.

\hspace*{-0.25in} 2003 - 2004,

Designed and programmed the SOIL software for the Structural Engineering Society of New Zealand (SESOC).

%\newpage
\vspace{0.15in} 

\section{EDUCATION}
\vspace{0.1in} 
Software Education Associates Ltd,\\
Advanced C++ Programming, 8-10 June 2009

The University of Auckland,\\
Master of Science with First Class Honours,\\
in Computer Science, 2004 
 
The University of Auckland,\\
Bachelor of Technology with First Class Honours,\\
in Information Technology, 2000-2003 
 
Lynfield College, 1998-1999 

\section{PUBLICATIONS} 
\vspace{0.1in}

"Improved Billboard Clouds for Extreme Model Simplification". In-Tai Huang, Kevin Novins and Burkhard Wuensche, \textit{in Proceedings of IVCNZ '04}, Akaroa, New Zealand, 21-23 November 2004, pp. 255-260. 

"Improved Billboard Clouds for Extreme Model Simplification". In-Tai Huang, MSc thesis, Department of Computer Science, University of Auckland, 2004.

\section{REFEREES} 
\vspace{0.1in}

%Jimmy Lin\\
%Programmer\\
%Right Hemisphere\\
%Ph: 021-2607786
Available upon request.
 
\end{resume}
\end{document}






























